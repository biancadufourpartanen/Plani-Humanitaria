\documentclass[]{article}

\usepackage{amsmath}
\usepackage[utf8]{inputenc}
\usepackage[T1]{fontenc}
\usepackage{graphicx}
\usepackage{float}

%opening
\title{Emergencia en Níger}
\author{Grupo 14: Bianca Dufour \& William Hedén}

\begin{document}

\maketitle

\noindent Se trata del problema de determinar el esquema óptimo de ayudar la emergencia en Níger en un dia.
\section{Conjuntos}
\begin{align*}
	i,j&: \text{Ciudad}, \quad i=1,\dots,7\\
	k&: \text{Tipo de vehiculo}, \quad j=1, 2
\end{align*}

\section{Parámetros}
\begin{align*}
	av_i &: \text{Ayuda disponible en ciudad i}\\
	d_i &: \text{Demanda de ayuda en ciudad j}\\
	cota_k &: \text{Cota superior de k}\\
	cap_k &: \text{Capacidad de ayuda de vehiculo k}\\
	velv_k &: \text{Velocidad vehículo k}\\
	cf_k &: \text{Un coste fijo para la conducción con el coche k por kilometro.}\\
	cv &: \text{Un coste fijo para mover una unidad de carga por kilometro.}\\
	velc_{i,j} &: \text{Velocidad máximo en el camino entre ciudad i y j.}\\
	vav_{k,i} &: \text{Número de vehiculos k disponibles en ciudad i.}\\
	dist_{i,j} &: \text{Distancia entre la ciudad i y j en kilometros.}\\
	budget &: \text{El coste total no puede superar el presupuesto.}\\
	qglobal &: \text{Este dia solo podemos enviar una carga total de qglobal.}
\end{align*}

\section{Variables}
\begin{align*}
		X_{i,j,k} &: \text{Número de vehículos k que van entre ciudad i y j.}\\
		carga_{i,j,k} &: \text{Cantidad de carga que van entre ciudad i y j con vehículo k.}\\
		Y_{i,j,k} &=
		\begin{cases}
		1 & \quad \text{si vehículo tipo k va de ciudad i a j.} \\
		0 & \quad \text{si no}\\
		\end{cases}\\
		load_i &: \text{Cantidad de carga que se queda en ciudad i.}\\
		Time_i &: \text{Tiempo en llegar a ciudad i.}\\ \\
		Coste &: \text{Coste total del ayuda a Níger.}\\
		Equidad &: \text{La carga que se queda en ciudad i dividido por la demanda de ciudad i.}\\
		Tiempo &: \text{El tiempo total del ayuda a Níger.}
\end{align*}

\section{Modelo}
Queremos hacer tres cosas en este modelo:
\begin{itemize}
	\item Minimizar el coste del ayuda total.
	\item Maximizar el equidad entre Agadez y Zinder.
	\item Minimizar el tiempo de hacer el operación.
\end{itemize}
Vamos a tratarlas una a una, y después resolvemos como una problema multiobjectivo por metas.

\begin{align*}
	\min \; & Coste = \sum_{i,j,k \mid dist_{i,j} > 0} dist_{i,j}\cdot (2\cdot X_{i,j,k}\cdot cf_{k}+cv\cdot carga_{i,j,k})\\
	\max \; & Equidad \leq  \frac{load_i}{d_i}, \quad \forall i \mid d_i > 0\\
	\min \; & Tiempo \geq Time_i, \quad \forall i \mid d_i > 0\\ \\
	\text{restricciónes:}&\\
	\forall j, \quad &\sum_{i,k \mid dist_{i,j} > 0} carga_{i,j,k} + av_j = \sum_{i,k \mid dist_{j,i} > 0}  carga_{j,i,k} + load_j\\
	\forall j,k, \quad &\sum_{i \mid dist_{i,j} > 0} X_{i,j,k} + vav_{k,j} \geq \sum_{i \mid dist_{j,i} > 0} X_{j,i,k}\\
	\forall j, \quad &load_j \leq d_j + av_j\\
	\forall j \mid d_j > 0, \quad &\sum_j load_j = qglobal\\
	\forall i,j,k \mid dist_{i,j} > 0, \quad & carga_{i,j,k} \leq cap_k \cdot X_{i,j,k}\\
	& Coste \leq budget\\
	\forall i,j,k \mid dist_{i,j} > 0, \quad & Time_j \geq Time_i + \frac{dist_{i,j}}{\min(velv_k, velc_{i,j})} - 10000\cdot (1-Y_{i,j,k})\\
	\forall i,j,k \mid dist_{i,j} > 0, \quad & X_{i,j,k} \leq cota_k \cdot Y_{i,j,k}
\end{align*}

\section{Solucion}
\begin{itemize}
	\item Minimizar el coste del ayuda total:\\ Coste =  65579.1667, Equidad = 0.2667, Tiempo = 127.2500 
	\item Maximizar el equidad entre Agadez y Zinder: \\Coste = 80000.0000, Equidad =  0.4762, Tiempo = 127.2500 
	\item Minimizar el tiempo de hacer el operación: \\Coste = 78018.7500, Equidad =  0.2667, Tiempo = 93.5000
\end{itemize}
\section{Codigo GAMS}

\end{document}
